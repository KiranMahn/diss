
\section{Specification \& Design}\label{process}

Describe all details of the  design and procedures used to achieve the project objectives. Do this chronologically. 


It should be detailed enough to allow for an assessment of the rigour of your process, and, in the case of research projects, in terms of how well grounded your research is in the research literature. In these cases, refer back to relevant sections in the previous chapter.

Say which  software lifecycle approach you used e.g., Waterfall, Spiral, Agile. 

How did you gather user requirements?

\subsection{Methodology}
Following research on the different available software lifecycle approach softwares and methods available, Kanban boards were chosen as they are easy to maintain for one person, help with workflow, and have been proven to work in a student setting as mentioned prior. GitLab Issue boards were selected as the platform to host the Kanban as each issue card can be linked to branches within repositories to keep track of which tasks and changes are for which branches to improve organisation. Labels were also added to signify the type of task for each issue card.
 
\subsection{Analysis}
How did you decide on the particular software artifact you decided to develop?

\subsection{Requirements}
Before beginning the process of gathering user requirements, an ethics form must be completed to ensure that the research follows ethical guidelines. To better understand user needs, a User Requirements survey was designed. In preparation, research was conducted through books on women’s experiences during perimenopause, including works by Kat Muir and Davina McCall, as well as reviewing research papers on health-tracking applications and a review of existing apps on the market to assess current solutions. Privacy emerged as a significant concern, with many apps engaging in surveillance capitalism and selling user data, often without informing the user. Only a handful of apps claim not to sell data, but even some app who claim not to sell data have been found guilty of making a profit off users data. To ensure a user-centered approach, User Stories were developed to capture potential users’ needs and expectations. As the project developed, user requirements and stories were continually adjusted to best reflect the goal of this project.

Here you explain what the functional and non-functional requirements are. Explain how you prioritised them. 
  See \url{https://www.nuclino.com/articles/functional-requirements} for more information. 

   \textbf{Functional requirement:} "The system must \textbf{\emph{do}} [requirement]."

    \textbf{Non-functional requirement:} "The system shall \textbf{\emph{be}} [requirement]."


Well-written functional requirements typically have the following characteristics:
\begin{description}



    \item[Necessary:] Although functional requirements may have different priority, every one of them needs to relate to a particular business goal or user requirement.

  \item[Concise:] Use simple and easy-to-understand language without any unnecessary jargon to prevent confusion or misinterpretations.

 \item[Attainable:] All requirements you include need to be realistic within the time and budget constraints set in the business requirements document.

 \item[Granular:] Do not try to combine many requirements within one. The more precise and granular your requirements are, the easier it is to manage them.

 \item[Consistent:] Make sure the requirements do not contradict each other and use consistent terminology.

 \item[Verifiable:] It should be possible to determine whether the requirement has been met at the end of the project.
    \end{description}

\subsubsection{Functional Requirements}
This is the \textbf{WHAT} of your artifact.

Functional requirements are product features that developers must implement to enable the users to achieve their goals. They define the basic system behavior under specific conditions. 

Functional requirements need to be clear, simple, and unambiguous.Examples:
\begin{itemize}
      \item Users must be able to log their period start and end dates to track cycle and period length.
      \item Users must be able to log their peri-menopausal symptoms and their severity daily to track changes over time.
      \item Users must be able to edit or delete logged symptoms and period data at any time.
      \item The app must store all user data locally using AsyncStorage on the users device, ensuring no data is stored on external servers.
      \item The app must include a calendar view where users can see logged symptoms and period data over time.
      \item The app must provide an option to reset user data to align with privacy-focused design principles
      \item The app must provice graph-based visualizations showing symptom frequency and period heaviness trends over time.
      \item The app must calculate and display the most common symptom based on user entries.
      \item The app must provide cycle length insights based on logged period data.
      \item The app must calculate average period length based on tracked cycles.
      \item Users must be able to access an Analysis Tab summarizing trends.
      \item The app must feature a Learn Page with information on perimenopause and related topics.
      \item Users must be able to access external links to trusted resources for more detailed information.
      \item The app must follow EU accessibility standards, including text scaling, color contrast, and screen reader compatibility.
      \item The app must support multiple languages to accommodate diverse users.
      \item Users must be able to enable or disable notifications/reminders for period tracking or symptom logging.
      \item The app must work fully offline, allowing users to track symptoms and view their data without an internet connection.
      
      
\end{itemize}


\subsubsection{Non-Functional Requirements}
This is the \textbf{HOW} of your artifact. Example non-functional requirement: ``When the submit button is pressed, the confirmation screen must load within 2 seconds.''

\subsection{Design}

\subsubsection{Interface Design}
Explain how you used wireframes, and how you tested these to design the user interface.


 \subsubsection{System Design}
 Show how you designed your database (if appropriate) and how you designed your system architecture, and the individual parts. Use UML and an Entity Relationship diagram

