
\section{Specification \& Design}\label{process}

Before beginning the process of gathering user requirements, an ethics form must be completed to ensure that the research follows ethical guidelines. To better understand user needs, a User Requirements survey was designed. In preparation, research was conducted through books on women’s experiences during perimenopause, including works by Kat Muir and Davina McCall, as well as reviewing research papers on health-tracking applications and a review of existing apps on the market to assess current solutions. Privacy emerged as a significant concern, with many apps engaging in surveillance capitalism and selling user data, often without informing the user. Only a handful of apps claim not to sell data, but even some app who claim not to sell data have been found guilty of making a profit off users data. To ensure a user-centered approach, User Stories were developed to capture potential users’ needs and expectations. As the project developed, user requirements and stories were continually adjusted to best reflect the goal of this project. The Design Testing Stage incorporates the System Usability Scale (SUS) evaluations and A/B Testing through microsoft forms. User interviews are also conducted to refine the design based on feedback. Recruitment for testing was also planned, stating the target audience for this study was women between the ages of 30 and 65.

\subsection{Methodology}
Following research on the different available software lifecycle approach softwares and methods available, Kanban boards were chosen as they are easy to maintain for one person, help with workflow, and have been proven to work in a student setting as mentioned prior. GitLab Issue boards were selected as the platform to host the Kanban as each issue card can be linked to branches within repositories to keep track of which tasks and changes are for which branches to improve organisation. Labels were also added to signify the type of task for each issue card.
 
\subsection{Analysis}
Given many peri-menopausal women are not receiving support from the government education system or their doctors, additional resouces and tools must be provided to help them navigate this stage of life. With the rise of technology, several tools are now available to allow women to track their peri-menopausal symptoms. However, these tools are often not user-friendly, do not provide enough educational information, or are not privacy-focused. The issue around privacy is especcially concerning as many apps are selling user data to third parties without user knoledge consent, and in todays political climate this can result in the incarceration of the user in some parts of the US. Since there are no apps that provide a comprehensive solution to these problems, the goal of this project is to create a user-friendly, educational, and privacy-focused peri-menopausal symptom tracking app.

\subsection{Requirements}

\subsubsection{Functional Requirements}
This Apps functional requirements were prioritized based on user needs and the project goal. Essential functional requirements impacting usability such as being able to navigate to a screen were prioritised over design and content details. The following functional requirements were identified:

\begin{itemize}
      \item Users must be able to log their period start and end dates to track cycle and period length.
      \item Users must be able to log their peri-menopausal symptoms and their severity daily to track changes over time.
      \item Users must be able to edit or delete logged symptoms and period data at any time.
      \item The app must store all user data locally using AsyncStorage on the users device, ensuring no data is stored on external servers.
      \item The app must include a calendar view where users can see logged symptoms and period data over time.
      \item The app must provide an option to reset user data to align with privacy-focused design principles
      \item The app must provice graph-based visualizations showing symptom frequency and period heaviness trends over time.
      \item The app must calculate and display the most common symptom based on user entries.
      \item The app must provide cycle length insights based on logged period data.
      \item The app must calculate average period length based on tracked cycles.
      \item Users must be able to access an Analysis Tab summarizing trends.
      \item The app must feature a Learn Page with information on perimenopause and related topics.
      \item Users must be able to access external links to trusted resources for more detailed information.
      \item The app must follow EU accessibility standards, including text scaling, color contrast, and screen reader compatibility.
      \item The app must support multiple languages to accommodate diverse users.
      \item Users must be able to enable or disable notifications/reminders for period tracking or symptom logging.
      \item The app must work fully offline, allowing users to track symptoms and view their data without an internet connection.
      \item The app must not crash or freeze during normal usage.
\end{itemize}

\subsubsection{Non-Functional Requirements}
\begin{itemize}
  \item The app should be easy to use and navigate, with a clean and simple design.
  \item The app home page should load within 3 seconds.
  \item All data visualization such as graphs, calendar, and analysis charts should render in under 2 seconds.
  \item The app should allow easy localization to support multiple languages.
  \item The UI should offer a dark mode and high-contrast mode to improve readability for all users.
  \item All text elements must support dynamic font resizing based on user preferences.
  \item The app should look the same for various screen sizes, including tablets and smaller phones.
  \item The app must be designed with easy language switching to support multiple languages in the future.
  \item The app should maintain consistent navigation and UI patterns across all features to reduce confusion.
\end{itemize}

\subsection{Design}

\subsubsection{Interface Design}
At the beginning of the design process, 6 pages were drawn by hand to rough sketch the layout of the app. Once the rough sketches were completed, they were digitized using Figma. The design was created with a focus on simplicity and ease of use. The app was designed to be user-friendly and intuitive, with a clean and simple design. The color scheme was chosen to be calming and easy on the eyes, with a focus on blues and greens. The app was then compared to the EU WAG 2.1 accessibility standards and adjusted to be accessible to all users, with large text and color contrast. 

\subsubsection{System Design}
In the interest of protecting user privacy, there is no database and all data is stored on the users local device. The data that is saved in AsyncStorage on their device is formatted as featured in the table below.

  \begin{table}[h!!]
    \caption{Structure User Data is Stored in AsyncStorage}
    \label{table:user-data}
    \begin{tabular}{llll}
    \hline
    Anonymous User &        &          &          \\ \hline
                  & Date1  &          &          \\
                  &        & Symptom1 & Severity \\
                  &        & Symptom2 & Severity \\
                  & Date2  &          &          \\
                  &        & Symptom1 & Severity \\
                  &        & Symptom2 & Severity \\
                  & Date n &          &          \\
                  &        & Symptom1 & Severity \\
                  &        & Symptom2 & Severity \\ \hline
    \end{tabular}
    \end{table}