\section{Conclusion}\label{conc}

By interpreting the results of this study, it was shown that there is a gap in the market for private and discreet focused period and symptom tracking apps. By creating a usable app with real data privacy and anonymity the aim and goals of this study have been accomplished. The app has evolved and improved with each round of user review and feedback which is reinforced by the rating scores and statists collected. Research showed that many women prefer to track symptoms manually but from the background research it is a likely possibility that this is due to lack of privacy and the sensitivity of data collected. It also may likely be due to cultural attitudes that do not encourage this sort of information to be discussed and reviewed.
