\section{Background Literature}\label{backg}

\subsection{Perimenopause}
The American Journal of Epidemiology\cite{Brambilla1994} states that perimenopause is the transition period of a womans life, which starts with a natural shift in ovulation and menstruation patterns and/or increased symptoms, and ends when a woman enters menopause. It classifies menopause as when a woman has not had her period for a year. Thie Lancashire and South Cumbria NHS Foundation Trust's article on Perimenopause, Menoause, and Pain\cite{LSCFT2024} states that during perimenopause, the body's production of estrogen, testosterone, and progesterone fluctuates significantly and can stay low forever if no treatment is taken. This change in hormones drastically changes the way a womans body and mind work. A Swiss Perimenopause study\cite{Willi2021} found that women experiencing lower estrogen and progesterone levels had higher suicide intent scores and were more likely to develop depression, which is backed up by another study from the Journal of Psychiatric Research which found a correlation between low progesterone states and suicide\cite{BACAGARCIA2010209}. Other effects of perimenopause are much more common, and perimenopause may not be the obvious source. During Perimenopause, 80\% of women experience hot flashes\cite{Bansal2019}, 77\% joint pain\cite{ScienceDaily2013}, 60\% memory issues\cite{Gaytri2018}, over 20\% experience heart palpitations\cite{Sheng2021}, 1/4 women have really heavy periods\cite{Harlow2011}, 50\% of women say it negatively impacts their sex lives\cite{BMS2016}, and 1/10 women leave their jobs because of menopausal symptoms\cite{Brewis2017}. It therefore comes as no surprise that almost 90\% of women to seek out their healthcare provider for advice on how to cope\cite{Guthrie2003}. However in the US, 3/4 of women who ask for medical help are left untreated causing women to turn to other sources of help and information\cite{Wolff2018}. 

\subsection{Menopause Education}