\section{Background Literature}\label{backg}

\subsection{Perimenopause}
The American Journal of Epidemiology\cite{Brambilla1994} states that perimenopause is the transition period of a womans life, which starts with a natural shift in ovulation and menstruation patterns and/or increased symptoms, and ends when a woman enters menopause. It classifies menopause as when a woman has not had her period for a year. Thie Lancashire and South Cumbria NHS Foundation Trust's article on Perimenopause, Menoause, and Pain\cite{LSCFT2024} states that during perimenopause, the body's production of estrogen, testosterone, and progesterone fluctuates significantly and can stay low forever if no treatment is taken. This change in hormones drastically changes the way a womans body and mind work. A Swiss Perimenopause study\cite{Willi2021} found that women experiencing lower estrogen and progesterone levels had higher suicide intent scores and were more likely to develop depression, which is backed up by another study from the Journal of Psychiatric Research which found a correlation between low progesterone states and suicide\cite{BACAGARCIA2010209}. Other effects of perimenopause are much more common, and perimenopause may not be the obvious source. During Perimenopause, 80\% of women experience hot flashes\cite{Bansal2019}, 77\% joint pain\cite{ScienceDaily2013}, 60\% memory issues\cite{Gaytri2018}, over 20\% experience heart palpitations\cite{Sheng2021}, 1/4 women have really heavy periods\cite{Harlow2011}, 50\% of women say it negatively impacts their sex lives\cite{BMS2016}, and 1/10 women leave their jobs because of menopausal symptoms\cite{Brewis2017}. It therefore comes as no surprise that almost 90\% of women to seek out their healthcare provider for advice on how to cope\cite{Guthrie2003}. However in the US, 3/4 of women who ask for medical help are left untreated causing women to turn to other sources of help and information\cite{Wolff2018}. 

\subsection{Menopause Education}
A University College London publication on women's post reproductive healt\cite{Aljumah2023} explores the extent of knowledge women have about menopause. When it comes to menopause, most women are left untreated and unsupported. Without sufficient education, most are left suffering due to hormonal imbalance and lifestyle changes they are unprepared for and do not have the information they need to help cope. In this study of 829 postmenopausal women, 90\% were never educated about the menopause. It is rarely included in sexual eduaction received in school and though awareness is increasing, it is still rarely talked about in the media and considered a taboo and private subject only to be talked about with your doctor\cite{Muir2022}. This is further problematic as doctors too are not well educated on menopause\cite{MenopauseSupport2021}. The NHS site on the treatment of menopause\cite{NHS2022} states that many menopause symptoms can be effectively treated with hormone replacement therapy (HRT), even symptoms such as hot flushes can improve within a few weeks. However, a study of 3000 british menopausal women who complained to their doctors of low mood or anxiety symptoms found that 66\% were offered anti-depressants instread of hormones\cite{NewsonHealth2019}. In fact, 1/4 of women are on anti-depresseants post menopause\cite{Brody2020} despite the fact that antidepressants don't help low mood in menopausal women with hormone imablances, and according to the National Institute for Health and Care Excellence\cite{NICE2019} HRT should be offered first. This may be largely due to lack of education doctors receive about menopause. 41\% of UK medical schools do not give any mandatory menopause education\cite{MenopauseSupport2021}. Professor Joyce Harper, an internationally renowned, award-winning scientist and a professor of reproductive science at UCL, states that “The data shows that women have a lack of education about this key life stage. Together with a reported lack of education from their healthcare professionals, women may be left undiagnosed and unsupported”\cite{UCL2023}
