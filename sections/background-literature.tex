\section{Background Literature}\label{backg}

\subsection{Perimenopause}
The American Journal of Epidemiology\cite{Brambilla1994} states that perimenopause is the transition period of a woman's life, which starts with a natural shift in ovulation and menstruation patterns and/or increased symptoms, and ends when a woman enters menopause. It classifies menopause as when a woman has not had her period for a year. This Lancashire and South Cumbria NHS Foundation Trust's article on Perimenopause, Menopause, and Pain\cite{LSCFT2024} states that during perimenopause, the body's production of estrogen, testosterone, and progesterone fluctuates significantly and can stay low forever if no treatment is taken. This change in hormones drastically changes the way a woman's body and mind work. A Swiss Perimenopause study\cite{Willi2021} found that women experiencing lower estrogen and progesterone levels had higher suicide intent scores and were more likely to develop depression, which is backed up by another study from the Journal of Psychiatric Research which found a correlation between low progesterone states and suicide\cite{BACAGARCIA2010209}. Other effects of perimenopause are much more common, and perimenopause may not be the obvious source. During Perimenopause, 80\% of women experience hot flashes\cite{Bansal2019}, 77\% joint pain\cite{ScienceDaily2013}, 60\% memory issues\cite{Gaytri2018}, over 20\% experience heart palpitations\cite{Sheng2021}, 1/4 women have really heavy periods\cite{Harlow2011}, 50\% of women say it negatively impacts their sex lives\cite{BMS2016}, and 1/10 women leave their jobs because of menopausal symptoms\cite{Brewis2017}. It therefore comes as no surprise that almost 90\% of women seek out their healthcare provider for advice on how to cope\cite{Guthrie2003}. However, in the US, 3/4 of women who ask for medical help are left untreated, causing women to turn to other sources of help and information\cite{Wolff2018}. 

\subsection{Menopause Education}
A University College London publication on women's post reproductive health\cite{Aljumah2023} explores the extent of knowledge women have about menopause. When it comes to menopause, most women are left untreated and unsupported. Without sufficient education, most are left suffering due to hormonal imbalance and lifestyle changes they are unprepared for and do not have the information they need to help them cope. In this study of 829 postmenopausal women, 90\% were never educated about the menopause. It is rarely included in sexual education received in school, and though awareness is increasing, it is still rarely talked about in the media and considered a taboo and private subject only to be talked about with your doctor\cite{Muir2022}. Talking to the doctor is also problematic as doctors are also not well educated on menopause\cite{MenopauseSupport2021}. The NHS site on the treatment of menopause\cite{NHS2022} states that many menopause symptoms can be effectively treated with hormone replacement therapy (HRT), even symptoms such as hot flushes can improve within a few weeks. However, a study of 3000 british menopausal women who complained to their doctors of low mood or anxiety symptoms found that 66\% were offered anti-depresseants instead of hormones\cite{NewsonHealth2019}. In fact, 1/4 of women are on anti-depresseants post menopause\cite{Brody2020} despite the fact that antidepressants don't help low mood in menopausal women with hormone imbalances, and according to the National Institute for Health and Care Excellence\cite{NICE2019} HRT should be offered first. This may be largely due to lack of education doctors receive about menopause. 41\% of UK medical schools do not give any mandatory menopause education\cite{MenopauseSupport2021}. Professor Joyce Harper, an internationally renowned, award-winning scientist and a professor of reproductive science at UCL, states that “The data shows that women have a lack of education about this key life stage. Together with a reported lack of education from their healthcare professionals, women may be left undiagnosed and unsupported”\cite{UCL2023}.

\subsection{Symptom Tracking Apps and Technology}
When women do not receive education from school, their communities, or their doctors about peri-menopause, they will look for tools and education from other sources such as websites or apps to research, track, and analyse their experience. Over 50 million women worldwide use apps to track their menstrual cycle and examine a variety of other cycle-related factors\cite{Kelly2023}. There are 300 menstrual tracking applications available for download and an estimated 200 million downloads worldwide\cite{Eschler2019}. Symptom monitoring and appraisal methods are effective for reducing menopausal symptoms, and improving health awareness, shared decision-making, patient-doctor communication, and treatment goal setting\cite{Andrews2021}. 

While these apps can be effective tools for dealing with symptoms, one of the most prominent issues with these apps is the lack of privacy. The apps often earn profits by selling users’ data to third parties, even if there is a promise of privacy advertised by the companies\cite{Gilman2021}. An article by the Director of Research for Sexual and Reproductive Health and Rights, Population Institute in Washington, DC titled, “Missed period? The significance of period-tracking applications in a post-Roe America”\cite{Kelly2023} highlights the increased concerns around this surveillance capitalism since the June 2022 overturning of Roe vs Wade in the US stated that the right to abortion is not constitutionally protected\cite{CoenSanchez2022}. It further explores how users personal tracking data may be used against them in court as evidence of having an abortion regardless of miscarriages, irregularities in menstrual cycles, and/or imperfect engagement with a period-tracking app. Some apps have even gone on record to say they will hand over users data to law enforcement if asked. The article even explains how some experts advise people who menstruate to track their periods on paper as opposed to using an app for their own protection. These `FemTech' mobile apps currently fall outside of the scope of the Health Insurance Portability and Accountability Act, which protects sensitive health information from being disclosed by covered entities without the patient’s consent or knowledge\cite{OCR2022}. This highlights the ever increasing need for privacy in menstrual tracking apps. 

A study\cite{Chan2019} exploring the design experience of digital period trackers found that to best design digital period trackers for users, Hertzum’s images of universal, situational and cultural usability should be used. This correlates with Dawsons concepts of evidence-based, usable, readable, interactive, and culturally sensitive design choices for health apps \cite{Dawson2020}. It found that a good period tracking app should know the users life stage, medical “situation”, contraception, purpose of tracking, and tracking interests. It also highlights the need for education resources within these health apps and the importance of users having access to relevant, reliable health information such as including external links to information. 

Not only do these peri-menopause apps have to be free, private, and personalised, but they must be accessible to all who want to use them. The European Accessibility Act (EAA) becomes law on the 28th of June 2025. The EAA is a landmark legal change that will improve the lives of disabled people by ensuring equal access to digital products and services for European Union (EU) consumers\cite{AbilityNet2024}. The EAA requires products and services to be Perceivable, Operable, Understandable, and Robust (known as POUR). 

There are many app rating frameworks available including the Mobile Application Rating Scale (MARS), A-MARS, THESIS, App Quality Assessment tool for Health-Related Apps (AQUA), the Digital Health Scorecard, the American Psychiatric Society App Evaluation Model, Beacon, Psyberguide, Happtique Health App Certification, Intercontinental Medical Statistics (IMS) Score, the EU Kitemark, and the FDA’s procedures for approval of digital tools as medical devices\cite{Morley2024}. MARS is designed to classifying and assessing the quality of mobile health apps\cite{Stoyanov2015}. In a studying reviewing many different mobile health app rating scales, MARS is one of the most popular scales and is one of the only scales designed to be used by the public or those with little to no experience reviewing health apps\cite{AzadKhaneghah2020}. Since MARS is widely known, easy to use, and assesses apps based on engagement, functionality, aesthetics, and quality of information, MARS was ultimately the best fit for this project.

%% Susan: 
%% Suggest rewrite: I concluded that Clue and Flo were the apps with the best designs, as theirs were clear, intuitive and fast. 

\begin{table}[h!!]
    \tiny
        \begin{center}
        \caption{App Search Terms and Results in order of appearance on the Apple App Store.}
            \label{table:app-search-terms}
            \resizebox{\textwidth}{!}{

                \begin{tabular}{llll}
                \textbf{Terms} & Perimenopause Tracker           & Perimenopause                         & Menopause                       \\ \cline{2-4} 
                                    & Natural Cycles: Birth Control   & Clue Period \& Cycle Tracker          & Flo Period \& Pregnancy Tracker \\
                                    & Clue Period \& Cycle Tracker    & Perry - perimenopause Community       & John Hopkins Menopause Guide    \\
                                    & Health \& Her Menopause App     & Health \& Her Menopause App           & Health \& Her Menopause App     \\
                                    & MenoLife - Menopause Tracker    & Caria: Menopause \& Midlife           & MenoLife - Menopause Tracker    \\
                                    & Balance - Menopause Support     & MenoLife - Menopause Tracker          & Caria: Menopause \& Midlife     \\
                                    & Perry - perimenopause Community & Clue Period \& Cycle Tracker          & Menopause Stage - Clearblue me  \\
                                    & Caria: Menopause \& Midlife     & Balance - Menopause Support           & Balance - Menopause Support     \\
                                    & Flo Period \& Pregnancy Tracker & Flo Period \& Pregnancy Tracker       & Joylux Menopausal Health App    \\
                                    & Period Tracker by GP Apps       & Joylux Menopausal Health App          & Clue Period \& Cycle Tracker    \\
                                    & Moody Month: Cycle Tracker      & Stardust: Period \& Pregnancy Tracker & ACOG                           
                \end{tabular}
            }
        \end{center}
  \end{table}

In order to decide what apps to rate, the top 10 results in the Apple App Store for the search terms “Perimenopause Tracker”, “Perimenopause”, and “Menopause” were listed (See table \ref{table:app-search-terms}). These terms were then ranked by frequency of appearance across these search terms and a final ten apps were chosen to be rated using the MARS system. From the ratings (See table \ref{table:MARS-app-review}), it was concluded that Clue and Flo had the best app design as their apps was clear, intuitive, and fast. Balance and Health and Her were more tailored to menopause and had significantly more perimenopause features including classification quizzes to determine menopause stage, pop ups and notifications when changes in patterns are detected, and resources to learn more about perimenopause and menopause. 

\begin{table}[h!!]
    \tiny
    \caption{MARS Quality Scores for Perimenopause Apps}
    \label{table:MARS-app-review} 
    \begin{tabular}{lrrrr
    >{\columncolor[HTML]{EFEFEF}}r 
    >{\columncolor[HTML]{EFEFEF}}r }
    \hline
    \textbf{App Name}            & \multicolumn{1}{l}{\textbf{Engagement}} & \multicolumn{1}{l}{\textbf{Functionality}} & \multicolumn{1}{l}{\textbf{Aesthetics}} & \multicolumn{1}{l}{\textbf{Information}} & \multicolumn{1}{l}{\cellcolor[HTML]{EFEFEF}\textbf{App Quality}} & \multicolumn{1}{l}{\cellcolor[HTML]{EFEFEF}\textbf{Subjective}} \\ \hline
    Clue                         & 3.6                                     & 4.5                                        & 4.3                                     & 3.6                                      & 4                                                                & 3.75                                                            \\
    Balance                      & 4                                       & 3.5                                        & 3.66                                    & 3.5                                      & 3.66                                                             & 3.75                                                            \\
    Caria                        & 2.8                                     & 3.5                                        & 4.33                                    & 3                                        & 3.4                                                              & 1.5                                                             \\
    Health and Her               & 3.8                                     & 4.25                                       & 3.66                                    & 3.16                                     & 3.71                                                             & 3                                                               \\
    Perry                        & 2.6                                     & 2.5                                        & 2                                       & 2.1                                      & 2.3                                                              & 1                                                               \\
    Natural Cycles               & 2.6                                     & 4                                          & 3                                       & 3.1                                      & 3.17                                                             & 1.25                                                            \\
    Flo                          & 3.8                                     & 4.25                                       & 5                                       & 3.5                                      & 4.13                                                             & 2.5                                                             \\
    JoyLux                       & 2.6                                     & 5                                          & 3.6                                     & 4.1                                      & 3.8                                                              & 1.75                                                            \\
    MenoLife                     & 0                                       & 0                                          & 0                                       & 0                                        & 0                                                                & 0                                                               \\
    John Hopkins Menopause Guide & 1.6                                     & 5                                          & 2.6                                     & 5                                        & 3.55                                                             & 2.5                                                             \\ \hline
    \end{tabular}
    \end{table}

\subsection{Project Direction}

Given the issues identified above, this project aims to address them by developing an app which allows women to track, analyse, and learn about peri-menopause while ensuring a focus on users privacy and anonymity. 


The objective of this research is to determine if a symptom tracking app is considered helpful to perimenopausal women to ease their symptoms and create an awareness of perimenopause while maintaining their privacy. 
%% Susan: 
%% Add sentence leading into next section: stating what you are going to do to deliver... 

