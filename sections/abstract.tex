
\begin{abstract}

As the fields of medical technology and digital health advances, more can be done to support peri-menopausal women through this life transition. Without proper education for doctors or the public, the everyday woman increasingly turns to technology to track her symptoms and take care of herself. However, FemTech corporations are using surveillance capitalism to sell these women's personal information which can have extremely serious ramifications for the women whose data is sold. During this study, research is done to determine the best way to help these women, and to develop a perimenopausal symptom tracking app that focuses on privacy and education. The app is designed to be user-friendly and intuitive, with a clean and simple design. The app has been evaluated using the System Usability Scale (SUS) and the Mobile App Rating Scale (MARS) to determine its usability and effectiveness. The app has been compared to other apps on the market to determine its strengths and weaknesses. The results of this study show that the app is usable and effective, but could in order to be most helpful to perimenopausal women a classification quiz to determine which stage of menopause the user is in should be added along with more detailed analysis of user data, and more learn resources.

\end{abstract}



