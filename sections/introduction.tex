\section{Introduction}

%% Susan:
%% While what is here is well-written, 
%% given that the introduction will usually outline the structure of the rest of the report, 
%% I suggest a little more explicit signaling of what is to follow e.g. 
%% This project aims to answer these questions by evaluating existing menopause apps, identifying areas for improvement (particularly in data privacy), and addressing them by creating a user-friendly... 
%%  
The perimenopause is an ill-defined time period that surrounds the final years of a woman's reproductive life\cite{PMC4834516}. Perimenopause is a transitional phase that is not well understood by many individuals including doctors, leading to a lack of awareness and knowledge about the symptoms and management strategies. Due to this lack of education and awareness, many individuals do not realise that the psychological, hormonal, and physical changes they are experiencing are due to perimenopause, leading to confusion and frustration\cite{Muir2022}. There are many digital tools to support those currently in menopause, but there is a gap in the market for digital tools that support individuals who are entering the perimenopause phase of life. What makes these apps and tools useful to a perimenopausal women and what are the key features that are needed to support them during this time? This project aims to answer these questions by creating a user-friendly, educational, and privacy-focused perimenopausal symptom tracking app.
