\section{Discussion \& Reflection}

\subsection{Interpreting the Results}
The results of this study show that there is a gap in the market for private and discreet focused period and symptom tracking apps. By creating a usable app with real data privacy and anonymity the aim and goals of this study have been accomplished. The app has evolved and improved with each round of user review and feedback which is reinforced by the rating scores and statists collected. 

Research showed that many women prefer to track symptoms manually but from the background research it is a likely possibility that this is due to lack of privacy and the sensitivity of data collected. It also may likely be due to cultural attitudes that do not encourage this sort of information to be discussed and reviewed.

\subsection{Reflection}
Trying different implementation strategies, while worthwhile, meant much time was wasted switching between different languages and frameworks. This was a valuable learning experience, but it did slow down the project. Future projects should have a more clear implementation plan from the start to avoid this issue.

\subsection{Challenges}
Version control practices became lax at one point during the project which had repercussions on the project timeline. Future projects should have a more strict version control policy. 

\subsection{Limitations}
The small number of participants did make the data collected less reliable and not necessarily representative. The app was also not fully tested on all devices and platforms, so there may be some bugs or inconsistencies that were not caught.

\subsection{Future Work}
Transitioning the application to use only typescript instead of a combination of typescript and javascript would be a good next step. This would allow for more strict type checking and reduce runtime errors. While researching, many good points were discovered about Expo router instead of React Navigation. While React Navigation did not pose and major issues and was easy to use, Expo Router may prove to be more sustainable for long term development and should be considered. Android support should also be added.

More analysis data should be added to the analysis page to provide the user with possible triggers or which symptoms are commonly tracked together or to notify the user when an unusual pattern is detected. The app could also be expanded to include more detailed analysis of user data, such as predicting when a user's next period will start based on their previous data, or include a quiz to indicate what stage of menopause they may be in and why. The app could be expanded to include more educational resources on perimenopause and related topics, such as articles, videos, and podcasts. 

A database connection could also be added so that those who want to back up their data can do so. Integrating the app with Apple Health or health watches and other apps to reduce the amount of work a user has to go through to input data may make the app more popular and give the app more data to increase analysis accuracy. Creating compatibility with the NHS digital front door app that is currently in the works would also be a good next step to increase the app's reach and usability. This could be accomplished in the form of easy data exporting from the tracking app to the NHS app so doctors are more aware of what their patients are going through and are better equipped to help perimenopausal individuals.
