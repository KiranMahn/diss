\section{Discussion \& Reflection}

\subsection{Interpreting the Results}

Here you will discuss your findings. This is especially relevant for research projects. You might 
 interpret what the data and evaluation implies, both for future research and for practice (if appropriate). 
 
 The discussion is \textbf{not} a review of literature. You should try to compare research findings with previous work,
provide  explanations for your findings,
discuss  research findings, in terms of their contribution.

%% Susan: 
%% Relate your reflections back to the original goals!
%% 
%% A usable app with real data privacy were two of the goal that immediately come to mind.
%% You should be able to demonstrate that your app is usable, based on Results
%% You can confidently state that you deliver real data privacy
%% because you explicitly designed the app to store data locally only.
%% 
%% And you could hypothesise that, given your background research, 
%% it is likely that women prefer to write down their symptoms manually 
%% because of the lack of data privacy and because of the sensitivity of the data
%% and because cultural attitudes to not encourage this sort of information 
%% to be discussed and reviewed.

The initial research shows many women prefer to write down their symptoms manually instead of using an app, but were hesitant to specify why. 

While this may be a helpful symptom tracking app to show data to your doctor, this will not make a big impact on the overall knowledge and awareness of peri-menopause and the struggle that perimenopausal women go through during this transition. Better education in school about menopause and perimenopause along with mandatory and detailed education for doctors would be a better and more effective solution to the problem.

\subsection{Reflection}
Trying different implementation strategies, while worthwhile, meant much time was wasted switching between different languages and frameworks. This was a valuable learning experience, but it did slow down the project. Future projects should have a more clear implementation plan from the start to avoid this issue.

\subsection{Challenges}
Version control practices became lax at one point during the project which had repercussions on the project timeline. Future projects should have a more strict version control policy. 

\subsection{Limitations}
The small number of participants did make the data collected less reliable and not necessarily representative. The app was also not fully tested on all devices and platforms, so there may be some bugs or inconsistencies that were not caught.

%% You may want to break this paragraph up into multiple paragraphs, 
%% splitting up technical, usability, analytics.
\subsection{Future Work}
Transitioning the application to use only typescript instead of a combination of typescript and javascript would be a good next step. This would allow for more strict type checking and reduce runtime errors. While researching, many good points were discovered about Expo router instead of React Navigation. While React Navigation did not pose and major issues and was easy to use, Expo Router may prove to be more sustainable for long term development and should be considered. More analysis data should be added to the analysis page to provide the user with possible triggers or which symptoms are commonly tracked together or to notify the user when an unusual pattern is detected. The app could also be expanded to include more detailed analysis of user data, such as predicting when a user's next period will start based on their previous data, or include a quiz to indicate what stage of menopause they may be in and why. Finally, the app could be expanded to include more educational resources on perimenopause and related topics, such as articles, videos, and podcasts. A database connection could also be added so that those who want to back up their data can do so. Integrating the app with Apple Health or health watches and other apps to reduce the amount of work a user has to go through to input data may make the app more popular and give the app more data to increase analysis accuracy. Creating compatibility with the NHS digital front door app that is currently in the works would also be a good next step to increase the app's reach and usability. This could be accomplished in the form of easy data exporting from the tracking app to the NHS app so doctors are more aware of what their patients are going through and are better equipped to help perimenopausal individuals.

